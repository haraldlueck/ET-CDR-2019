\FloatBarrier
\label{sec:optlayout}

\begin{wrapfigure}{r}{0.4\textwidth}
%\begin{figure}{H}
	\centering
	\includegraphics[width=0.3\textwidth]{Intro/Intro_Figures/NestedDetectors.pdf}
	\caption{Three nested detectors in a triangular arrangement will 
	form the final Einstein Telescope geometry.}
	\label{fig:NestedDetectors}
\end{wrapfigure}
In its final construction stage the Einstein Telescope will consist of three
nested detectors, which will be arranged in a triangular pattern as shown in
figure\,~\ref{fig:NestedDetectors}. In contrast to the traditional L-shaped
geometry of the first and second generations of gravitational wave detectors
this arrangement is equally sensitive for both polarisations of the
gravitational wave. Additionally it shows a more isotropic antenna pattern
compared to the L-shaped detectors.%, as shown in figure\,~\ref{fig:response}. 
The
overall frequency range covered will reach from a few Hertz to about 10\,kHz.

Each individual detector in turn will comprise two interferometers, one
specialised for detecting low-frequency gravitational waves and the other one
for the high-frequency part of the spectrum. The sensitivity goal for each interferometer is
shown in figure\,\ref{fig:ET_sensitivity}. %\\ 
Each individual interferometer has a  dual-recycled Michelson topology with
Fabry-Perot arm cavities. This is a mature and well tested configuration
currently employed  second-generation detectors, such as Advanced LIGO and
Advanced Virgo. 

This section describes the details of the ET optical layout, such as the laser
beam sizes, beam shapes and distances between optical components inside the arm
cavities and central interferometer including the power and signal recycling
cavities. A schematic sketch of the optical layout of all core optical of the
interferometers is shown in figure~\ref{Fig:Simple_ETv1}. 

\begin{figure}[p]
\centering
\includegraphics[width=0.9\textwidth]{Detector/Optics/OpticalLayout/OpticalLayoutFigures/ET_April2011_v2.png}
\caption{\tcb{PICTURE TO BE REPLACED BY UPDATED LAYOUT. Will probably choose something with more details on the infrastructure  (tunnels, shafts etc) and less details on what is inside vacuum system} Simplified drawing of the low and high frequency core interferometers of a single ET-detector. Injection and detection optics as well as filter cavities have been omitted for clarity. Please note that the complete ET observatory consists of three such detectors.%
}
\label{Fig:Simple_ETv1}
\end{figure}

\subsubsection{A xylophone design for ET}\label{sec:xylophone} 
Spanning the detection band over four orders of magnitude in frequency, as is
asked for third-generation GW observatories such as ET, is technically extremely
challenging: different noise types dominate the various frequency bands and
often show opposite responses for different tuning of the same design parameter.

In the following we provide examples of fundamental issues of a broadband
third-generation interferometer that could be resolved by using a set of
xylophone detectors~\cite{Hild2010a}:
\begin{itemize}
\item \textbf{Control Noises}: Many noise sources limiting the second generation
GW detectors at the low frequency end seem to become more challenging with
increased optical power: classical radiation pressure forces and torques
originating from residual misalignments and beam jitter dominate the dynamics of
the interferometer mirrors and hence the local and global control loops. 
The xylophone concept will help ET to achieve its unprecedented low-frequency
sensitivity target, by minimising the radiation pressure driven forces on the
mirrors of the ET LF detector. 
%Similarly, thermal deformation of the main
%optics at high optical power are currently a challenge for achieving low 
%optical losses during operation. 
\item \textbf{Shot Noise vs Radiation Pressure Noise}: Due to the fact that the
shot noise contribution scales inverse with optical power, but the photon
radiation pressure noise contribution on the other hand scales proportional to
the optical power, it will be hard to obtain the desired bandwidth with a
single detector. Therefore, again it might be useful to split ET into a
low-power low-frequency and a high-power high-frequency companion.
\item \textbf{High Power vs Cryogenic Temperature}: In a single broadband ET
observatory the simultaneous use of high optical power (a few megawatts) to
achieve the required high frequency sensitivity and test masses at cryogenic
temperatures in the 10 to 20\,K range in order to provide the required
suppression of thermal noise would pose a strong challenge. Even though tiny,
the residual absorption of the dielectric mirror coatings deposits heat in the
mirrors which is difficult to extract, without spoiling the performance of the
seismic isolation systems. A possible solution for this problem would be to
build a xylophone observatory consisting of a high-frequency detector featuring
high power and high temperature, and a low-frequency detector featuring low power
and cryogenic temperatures.
\end{itemize}

The xylophone concept was first suggested for Advanced LIGO, proposing to
complement the standard broadband interferometers with an interferometer
optimized for lower frequency, thus enhancing the detection of high-mass binary
systems \cite{Shoemaker2001LIGOXylophone, Conforto2004}.
One may think that a xylophone might significantly increase the required
hardware and its cost by the need to build more than one broadband instrument.
However, such an argument does not take the technical simplifications that it
would allow, the better reliability of simpler instruments, and the more
extensive scientific reach allowable into account.

\begin{comment}
For example splitting a third-generation observatory into a low-power,
low-frequency  and a high-power high-frequency interferometer, has not only the
potential to resolve the above mentioned conflict of photon shot noise  and
photon radiation pressure noise, but also allows to avoid the combination of
high optical power and cryogenic test masses. To reduce thermal noise to an
acceptable level in the low frequency band, it is expected that cryogenic
suspensions and test masses are required. Even though tiny, the residual
absorption of the dielectric mirror coatings deposits a significant amount of
heat in the mirrors. Since this heat is difficult to extract, without spoiling
the performance of the seismic isolation systems, it imposes a limit on the
maximum circulating power of a cryogenic interferometer.
\end{comment}

\begin{figure}[thbp]
\centering
\includegraphics[width=0.8\textwidth]{Detector/Optics/OpticalLayout/OpticalLayoutFigures/Layout_overview.pdf}
\caption{Simplified sketch of the ET low and high frequency core interferometers of a single ET-detector.}
\label{Fig:opt_lay_over}
\end{figure}

\begin{table}
\begin{center}
\begin{tabular}{l l l}
\hline
\hline
Parameter & ET-D-HF   & ET-D-LF \\
\hline
Arm length & 10\,km & 10\,km \\
Input power (after IMC) & 500\,W & 3\,W \\
Arm power & 3\,MW & 18\,kW\\
Temperature & 290\,K &  10\,K  \\
Mirror material & fused silica & silicon \\
Mirror diameter / thickness & 62\,cm / 30\,cm & min 45\,cm/ \tcb{(TBD)} \\
Mirror masses & 200\,kg & 211\,kg \\
Laser wavelength & 1064\,nm & 1550\,nm \\
SR-phase & tuned (0.0) & detuned (0.6)\\
SR transmittance & 10\,\% & 20\,\% \\
Quantum noise suppression &  freq.\ dep.\ squeez.& freq.\ dep.\ squeez.\\
Filter cavities & $1 \times $300\,m  & $2 \times$ 1.0\,km \\
Squeezing level  & 10\,dB (effective) & 10\,dB (effective) \\
Beam shape & TEM$_{00}$ & TEM$_{00}$\\
Beam radius & 12.0\,cm & 9\,cm \\
Scatter loss per surface & 37.5\,ppm & 37.5\,ppm \\
Seismic isolation & SA, 8\,m tall & mod SA, 17\,m tall \\
Seismic (for $f>1$\,Hz) & $5\cdot 10^{-10}\,{\rm m}/f^2$ & $5\cdot 10^{-10}\,{\rm m}/f^2$  \\
Gravity gradient subtraction & none \tcb{(TBC)} & none \tcb{(TBC)}\\
\hline
\hline
\end{tabular}
\caption{Summary of the most important parameters of the ET-D high and low frequency interferometers. SA~=~super attenuator,  freq.\ dep.\ squeez.~=~squeezing with frequency dependent angle. \label{tab:summary14}}
\end{center}
\end{table}

The baseline for ET is a 2-band xylophone detector configuration, composed of a
low-frequency (ET-LF) and a high-frequency (ET-HF) detector. Both
interferometers are Michelson interferometers featuring 10\,km armlength and an
opening angle of 60 degrees.  Due to their similar geometry both detectors will
share a single facility. Table~\ref{tab:summary14} gives a brief overview of the
main parameters of the analysed low-frequency (ET-LF) and high-frequency (ET-HF)
detector. Figure~\ref{Fig:opt_lay_over}   shows sketches of the corresponding
core interferometers and the filter cavities. The full layout of the two core
interferometers of a single ET detector is depicted in
Figure~\ref{Fig:Simple_ETv1}.

% -------------------------------------------------------------------------------------------
\subsubsection{Arm cavity design}
\label{sec:arm_cavity_design}
The size and shape of the laser beam inside the interferometer is defined by the
surface shape of the cavity mirrors; the beam sizes at the arm cavity input
mirrors (IM) and arm cavity end mirrors (EM) as well as the position of the
cavity waist are determined by only two parameters, the radii of curvature (ROC)
of IM and EM. Since inside the two Fabry-Perot cavities of the Michelson
interferometer the GW interacts with the laser light, creating signal sidebands,
the two arm cavities can be seen as the heart of the ET interferometers. The
characteristics of the arm cavities have not only a high impact on the detector
sensitivity and bandwidth, but also on the overall detector performance.


The choice of the beam size on the arm cavity mirrors is a trade-off process
taking the following considerations into account:
\begin{itemize}
%  \item For a given cavity length there is a minimal achievable beam size, which
%  is determined by the divergence of the beam.
\item Above a minimal beam size, given by the length of the interferometer arm,
any further increase in beam size leads to a reduction of the various 
thermal noise contributions from the cavity mirrors.
\item The upper limit for the manageable beam size is given firstly by
the maximum available mirror substrate size and secondly by the approaching of
the cavity instability.
\end{itemize}

%\textbf{Arm cavity mirror size}
A common method to define the mirror size is to demand the optical power loss
due to clipping (light being lost because it `falls over the edge of the
mirror') to be less than $1\,$ppm. This results in a scaling factor between 
beam and mirror radius of 2.63 (for TEM$_{00}$ modes), see~\cite{Chelkowski2009}.

\begin{comment}
\begin{center}
\begin{tabular}{|l|c|c|}
	\hline
	mode  & TEM$_{00}$ & LG$_{33}$\\
	\hline
	mirror radius to beam  radius & 2.63 & 4.31\\
	\hline
\end{tabular}
\end{center}

\textbf{Minimal mirror sizes for ET}

Using the currently discussed options for ET we can compute minimal mirror sizes
for various options, by using $L=R_{C}$ resulting in $w_{\rm
min}=\sqrt{\frac{L\lambda}{\pi}}$.
\begin{center}
\begin{tabular}{|l|c|c|}
	\hline
setup & min beam radius  & min mirror diameter \\
         & [cm] & [cm] \\
	\hline
LG$_{33}$, 1064\,nm  &  5.8      &  50.2     \\
	\hline
TEM$_{00}$, 1550\,nm  &  7.0      &   37.0    \\
	\hline
\end{tabular}
\end{center}

\textbf{Realistic mirror sizes for ET}

The minimal beam size can be computed using $L=R_{C}$ resulting in 
$w_{\rm min}=\sqrt{\frac{L\lambda}{\pi}}$ as:
\begin{center}
\begin{tabular}{|l|c|c|}
	\hline
wavelength & min beam radius  & min mirror diameter \\
         & [cm] & [cm] \\
	\hline
1064\,nm  &  5.8      &  30.6     \\
	\hline
1550\,nm  &  7.0      &   37.0    \\
	\hline
\end{tabular}
\end{center}

\textbf{Realistic mirror sizes for ET}
\end{comment}

Using the minimal beam sizes is obviously not optimal in terms of thermal noise.
Therefore we intend to push the beam sizes for ET towards the maximum feasible
size, which corresponds to about 60\,cm substrate diameter for fused silica
mirrors and 45\,cm for the silicon mirrors. Assuming 10\,km long arm cavities,
we can derive the following arm cavity characteristics.

\begin{center}
\begin{tabular}{|c|c|c|c|c|c|c|c|c|}
  \hline
IFO & $\lambda$& beam & mirror $\varnothing$ & $R_{\rm C}$ & $w_0$ &$z_0$ & $w$ & $g-$factor \\
\hline
ET-HF & 1064\,nm & TEM$_{00}$ & 62\,cm & 5070\,m & 1.42\,cm & 5000\,m & 12.0\,cm & 0.95\\
\hline
ET-LF & 1550\,nm & TEM$_{00}$ & 45\,cm &5580\,m & 2.9\,cm & 5000\,m & 9.0\,cm & 0.63\\
\hline
\end{tabular}
\end{center}

% ----------------------------------------------------------------------------------------------------

\FloatBarrier
\subsubsection{Central interferometer design}
\label{sec:opt_layout_CITF}

The central interferometer consists of the two recycling cavities and the
central Michelson interferometer formed by the beam splitter and the arm cavity
input mirrors. The design of the central interferometer is mainly determined by
two constraints. First of all it should allow for the implementation of
non-degenerate recycling cavities. Second, the central interferometer has to
serve as mode-matching telescope for the arm cavities.

The non-degenerate recycling cavity design used by the advanced detectors 
%(see Figure~\ref{Fig:Sec_Optics_AdvLIGO_IFO_Schematic}) 
can probably not be directly
adapted for ET, because no beam splitter substrates of the required dimensions
would be available. For example the high frequency interferometer featuring an
opening angle of 60 degree would require a beam splitter with a diameter of
more than 120\,cm.

Therefore we plan to investigate design options making use of input mirror
substrates including a focussing lens and introducing a two telescope 
mirrors in a z-configuration between the IMs and the central beam
splitter, see~\ref{Fig:mode_matching_telescope}.
Please note that the arm cavity mirrors are \tcb{(comment nchriste:) and?'} possibly the telescope mirrors are
the only full sized optical elements and that the beam splitter and recycling
mirrors  can be significantly smaller.

\begin{figure}[thbp]
\centering
\includegraphics[width=0.4\textwidth]{Detector/Optics/OpticalLayout/OpticalLayoutFigures/telescope_drawing.pdf}
\caption{Simplified sketch of mode matching telescopes in the interferometer
arms.
\bluecomment{Replace this image with a cleaner sketch.}}
\label{Fig:mode_matching_telescope}
\end{figure}

\begin{comment}
\textbf{Layout option for TEM$_{00}$, 1550\,nm}
\nopagebreak

The optical parameters of a possible solution based on a arm cavity length of
$L=10$\,km and a TEM$_{00}$ mode at 1550\,nm are provided below:
\begin{itemize}
\item focussing element in or near the ITM with a focal length of $f=303$\,m
\item distance ITM--BS: 300\,m
\item distance BS--MPR: 10\,m
\item beam size on BS: 6\,mm
\item beam size on MPR: 3.4\,mm
%\item Rayleigh range in central interferometer: 6.3\,m
\end{itemize}

The recycling cavity formed by MPR and ITM has a length of 310\,m and a free
spectral range of 484\,kHz. The round-trip Gouy phase is given by $\approx
9.6$\,deg which corresponds to a mode separation frequency of $25.8$\,kHz.

\textbf{Layout option for LG$_{33}$, 1064\,nm}
\nopagebreak

Using the same distances and focussing elements for the interferometer with a
LG$_{33}$, 1064\,nm beam, we also obtain reasonable numbers:
\begin{itemize}
\item beam size on BS: 4.7\,mm
\item beam size on MPR: 2.7\,mm
%\item Rayleigh range in central interferometer: 6.7\,m
\item Gouy phase: 10.5\,deg
\item mode separation frequency:  28.1\,kHz
\end{itemize}

These layout options are not yet optimised but they show that a separation
between beam splitter and input optics in the order of 300\,m represents a
useful baseline. The numbers for the beam sizes at the beam splitter and
recycling mirrors in both cases need to be checked against a detailed thermal
noise computation.

Furthermore, the design needs to be evaluated for losses originating from
astigmatisms inside the recycling cavities as well as for scattered noise
issues.
\end{comment}




