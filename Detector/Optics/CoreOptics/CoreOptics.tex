This section focuses on the mirrors forming the arm cavities of the interferometer (IM and EM). Those optics, also called test masses, are the largest and most critical ones whose displacement noise can directly degrade the gravitational waves signals.

To ensure the best optics, the three ingredients of a mirror: substrate, polishing and coating will have to use state of the art technologies. An overview of current achievements and the core optics strategy for ET is presented in the following sections. The working temperature of the optics have a strong impact on the technological choices to be made. 

\subsubsection{The substrate materials}

The substrate of the large optics must meet some drastic specifications in term of optical and mechanical specifications, moreover it should be available in large sizes and it must be shaped to the atomic level. Due to such constraints, few specific materials can be considered: fused silica for room temperature interferometer and sapphire and silicon for cryogenic temperature.\\

\paragraph{Fused silica} is the substrate of choice for all the current room temperature interferometers.  Due to its extensive
use for first and second generation gravitational wave detectors, this material has been extensively characterised at room temperature. Moreover the polishing and coating are now well mastered for this material \cite{pinard2017mirrors}.

Fused silica is transparent in the visible and near infrared, the operating wavelength of room temperature ET will be 1064~nm, the same as advanced interferometers. At this wavelength fused silica exhibits very low bulk optical absorption (below 1 ppm/cm) with high homogeneity of refractive index (relative optical path length < 0.1 nm/cm over the central part) and very low birefringence (around 1 nm / cm). The material can be isotropic in 3 dimensions which is ideal for the central beansplitter where laser beams are crossing the substrate at different angles.

In addition to its outstanding optical properties, fused silica presents a very low Brownian thermal noise at room temperature. Additionally, there exist techniques to fabricate quasi-monolithic suspensions based on pulled fused silica fibres and silicate bonding to further reduce the suspension thermal noise. These techniques have demonstrated their reliability and performances for years in the GEO600 detector \cite{plissi1998aspects} and so have been successfully implemented in LIGO and Virgo detectors \cite{robertson2002quadruple,lorenzini2010monolithic}.

The upgrade of Advanced Virgo, called Advanced Virgo+ will use test masses in fused silica of diameter 55~cm weighting 100~kg, that would be similar to the optics for ET-HF which will be in the same material and with a diameter of 62~cm for 200~kg. Hence, fused silica as a large optical substrate presents the best performances at room temperature with a very limited risk.\\


\paragraph{Sapphire} is considered as a substrate candidate for ET-LF since as a monocrystal it is well suited for low temperature optics. Sapphire crystals have exceptional physical properties, impressive hardness (only exceeded by diamond), large optical transmission from visible to infrared and large resistances to thermal shocks and chemicals.

One crucial point about operating at low temperature is the constraint on the substrate optical absorption. The heat generated by the passing laser beam increases the test mass temperature and it can only be efficiently removed through the thin suspension fibers. Measuring the optical absorption at 1064~nm of several large  sapphire ingots has shown the disparity of the results, very low optical absorption ($<$ 50 ppm/cm) can never been granted and could only be achieved after selection of the best substrates. The current best sapphires were used as KAGRA input mirrors of the arm cavities, with a diameter of 22 cm for a thickness of 15~cm, they present an absorption of around 40~ppm/cm \cite{PhysRevD.89.062003}.  

To strongly reduce the birefringence, special care must be taken to ensure that the optical axis is well aligned with the c-axis of the crystal. Substrate refractive index inhomogeneity is larger than in fused silica but it can be compensated during the polishing process for transmissive optics. 

Low thermal noise can be expected from sapphire substrates at low temperature. The thermo-elastic noise is strongly reduced thanks to the decrease of the thermal expansion coefficient ($\alpha = 1 \times 10^{-9}$ [K$^{-1}$]\cite{taylor1996measurement} at 10~K) and a bulk mechanical loss of $ 5 \times 10^{-9}$ (10~K) was measured ensuring a low substrate Brownian thermal noise \cite{uchiyama1999mechanical}.

Sapphire is currently used in KAGRA the Japanese cryogenic interferometer and the experience acquired by operating such optics at low temperature is full of insights \cite{Akutsu_2019}. In particular, the issue of a growing thin layer of ice on the optics which increases the optical losses in the arm cavities was highlighted \cite{hasegawa2019snow} and the impact on thermal noise later estimated \cite{PhysRevResearch.1.013008}. This issue could be mitigated by lowering the pressure in the long arm vacuum pipes and increasing the cold vacuum duct (working as a cryotrap) in front of the mirrors. The necessity for the substrate of the mirrors to be precisely along the c-axis was also demonstrated as it could lead to detrimental birefringence and imperfect correction of the refractive index inhomogeneity, both synonymous of optical losses \cite{somiya2019birefringence}. 

The largest incertitude regarding sapphire substrate for ET-LF is the availability of 200~kg ingot with extremely good optical properties, in particular too high optical absorption (few tens of ppm/cm) could be a showstopper. Sapphire boules above 200~kg are currently produced (to be cut in wafer for LED substrates) \cite{Bigspapphire2019} but the overall optical quality and uniformity are still to be demonstrated considering the ET requirement.\\

\paragraph{Silicon} is the second candidate material for the test masses for the cryogenic interferometer ET-LF. Compared to fused silica and sapphire materials presented previously, silicon is not transparent at 1064~nm and so the operating wavelength of the detector has to be shifted to 1550~nm. 

Silicon has excellent mechanical and thermal properties and is easily available in relatively high quality due to the large market of the semiconductor industry. The coefficient of thermal expansion is zero at two special temperatures around 18~K and 125~K \cite{lyon1977siliconexpansion}. At these temperatures the contribution of thermo-elastic noise will therefore vanish. The mechanical loss of silicon has been studied by Q-factor measurements. It was experimentally shown that silicon bulk samples can reach mechanical losses as low as $1 \times 10^{-9}$ at 10~K. \cite{mcguigan1978siliconQ}.

The maximum available diameter and purity of silicon depends on the fabrication process. The two main growing processes for single crystal silicon used by the semiconductor industry are the Czochralski (CZ) and the Float Zone (FZ) methods. CZ silicon is grown from a silicon melt in a silica crucible. It results in
relative large samples with a reasonable purity. The most dominant impurities in undoped CZ-grown silicon are carbon (typically 10$^{-18}$ cm$^{-3}$) and oxygen (typically up to 10$^{-19}$ cm$^{-3}$). 

In contrast, FZ silicon contains also these impurities but in much smaller concentrations (up to 10$^{-3}$ times smaller). During the FZ growth process, single or poly-crystalline silicon is remelt by means of inductive heating in vacuum or under an inert atmosphere. Impurities dissolve better in the melt than in the solid part. The re-crystallised material has therefore a higher purity than the initial one. By slowly sweeping the melt from one end to the other it is possible to purify in steps. The mechanism of inductive heating sets limits to the currently available setups and leads to smaller currently available samples.

Using the CZ growth technique, silicon ingots up to 45~cm of diameter can be produced however 30~cm is still the dominant wafer diameter in the semiconductor industry. For FZ silicon the diameter is currently limited to 20~cm.

In the recent years, motivated by the possible use of silicon as test mass materials, the optical properties of silicon have been thoughtfully characterised. Regarding the bulk optical absorption at 1550~nm, it was demonstrated the direct link between impurities concentration and the absorption \cite{degallaix2013abs_silicon}. For FZ silicon, absorption below 5~ppm/cm has been measured which is compatible with the ET-LF requirement. During the absorption studies, excess optical absorption at the surface of silicon was reported \cite{khalaidovski2013indication} which is likely linked to the polishing techniques used and not intrinsic to the material \cite{bell2017Sisurf}. According to the latest measurement, magnetic Czochralski (mCz) growth technique would be the most suitable approach for ET as it can combine large diameter ingot (45~cm diameter) with very low impurities since absorption around 20~ppm/cm has been achieved on one sample.

Thanks to the very low intensity of the laser beam in ET-LF, non linear effects such as two photon absorption \cite{bristow2007two} or Kerr effect are expected to be negligible.

Other caracterisations done in the framework of the Einstein Telescope include the measurement of thermo-optic coefficient at low temperature \cite{komma2012SithermoOptic} which is essential to derive the thermal lensing magnitude and the substrate thermo-refractive noise and also the birefringence \cite{kruger2015birefringenceSi} which is in the same order of magnitude as for sapphire. 

Like for sapphire, the choice of silicon substrate for ET-LF will be validated when it could be demonstrated that this material can come in large size (diameter of 450~mm) together with a very low optical absorption (order of several ppm/cm) at 1550~nm. Silicon ingot made with mCz seems to meet those specifications on some sample but the repeatability has yet to be proven.

\subsubsection{Surface polishing achievement}

The polishing capability will depend on the substrate material. Polishing of fused silica is well mastered thanks to current generation of room temperature interferometers and hence presents little risks. For the Einstein Telescope, the same flatness and roughness that was achieved \footnote{flatness inferior to 0.5~nm RMS and roughness below 0.1~nm RMS on the central part \cite{pinard2017mirrors}.} for Advanced detectors should be enough albeit on a larger area since the ET mirrors will be bigger. Due to the heavier substrate, special handling tools will have to be manufactured but according to the polishing companies, there is no showstopper. The large end mirrors of Advanced Virgo+ with a diameter of 550~mm and weighting 100~kg will be a pertinent pathfinder before the procurement of the ET mirrors.

Sapphire which is one the hardest known material has always been very difficult to polish. Until recently, pushed by the mirror requirement from KAGRA, a surface quality as good as the one for fused silica has been demonstrated \cite{hirose2014sapphire}. This a tremendous achievement even at the price of a larger cost and longer delay compared to fused silica.


independent of temperature! what we have achieved right now, size, surface quality
Should not be an issue.
Silicon more brittle.

\subsubsection{The coating materials and design}

A quick reminder of the problem (the longer version is in the original study). Focus on the coating thermal noise
Formula, thermal noise depends on temperature, beam size, coating loss angle

focus on the latest progress (multi-materials approach, promising materials),
Planned strategy to achieve what we want for ET

Already Ok for room temperature ?

\subsubsection{Thermal noise curves}

noise curve in this section similar to figure 187 from original study, tell explicitly which parameters have been updated to derive the curves

\paragraph{LF interferometer}

\paragraph{HF interferometer}


\subsubsection{Other core optics}

Central interferometer optics such as Power and Signal Recycling Mirrors (PRM and SRM) or Beamsplitter will be smaller in size (diameter in the order of 10~cm) and at room temperature. Fused silica is hence the preferred material for the substrates. The coating will use the same materials as for ET-HF to benefit from state of the art deposition process. The procurement of those optics does not represent any challenges and will be straightforward.