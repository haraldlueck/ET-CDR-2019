\label{sec:lasers}

%\tcb{version 5 Nov 2019 B. Willke}

Each of the ET interferometers will use a high power laser system with low
intrinsic noise called the high power laser (HPL) in the following. As no
current laser design can meet the ET stability requirements several nested
stabilization control loops are required to  reduce the fluctuations of the
free-running laser before the light can be injected into the main
interferometer. Like in all currently operating GWDs a first ET laser
stabilization layer will be installed on a seismically isolated laser table
outside the vacuum system. The combined system of HPL and this stabilization
layer is called the prestabilized laser (PSL) and is discussed in this section.
We will start with a discussion of the PSL requirements for ET-LF and ET-HF
followed by a description of the HPLs for both interferometers. The last
subsection will discuss passive and active noise reduction concepts that will
prepare the light for injection into the suspended input modecleaner (IMC).
%described in the \tcb{auxilliary optics subsection} %\ref{sec:{Auxilliary
%Optics}}.

{\bf Requirements}\\
The ET-HF HPL has to operate in a single-frequency continuous-wave (cw) mode at
a wavelength of 1064\,nm and needs to deliver 700\,W in a linear polarized
fundamental spatial $\rm HG_{00}$ mode. With the assumption of roughly 30\% loss
between the laser and the IMC output this leaves 500\,W at the input of the main
interferometer. The higher order mode content of this laser should be below 10\%
and the polarization purity at least 1/10. The ET-LF HPL needs to operate in a
single-frequency continuous-wave (cw) mode at a wavelength of approximately
1550\,nm with similar spatial and polarization purities as the ET-HP HPL. A
laser power of 5\,W is required to allow for 3\,W to be injected into the main
interferometer. 

Both HPLs need to provide actuators with sufficient dynamic range and speed to
allow for a suppression of their free running laser frequency-, power- and
pointing noise and to compensate for noise introduced between the PSL interface
on the laser table and the main interferometer's reference frame. The relative
power noise (RPN) in the megahertz frequency range should be shot noise limited
for 100\,mA photo current. In addition the HPLs need to operate reliable  with
small drifts and only limited maintenance requirements. 

Without a detailed interferometer design only a rough estimation of the ET PSL
noise requirements can be given. The frequency noise of the light injected into
the IMC will be subject to Doppler noise between the laser table and the
suspended reference frame. A frequency noise in the $10\,\rm{mHz} /
\sqrt{\rm{Hz}}$ range should be well below this level and hence adequate as PSL
requirement. Following the same line of thought the beam pointing requirement of
the ET PSLs will be similar to the one of advanced GWDs with relative lateral
and angular beam fluctuations in the $10^{-6} / \sqrt{\rm{Hz}}$
range\cite{Kwee2012}. With a similar power noise coupling and a 10 fold improved
sensitivity compared to the advanced GWDs the ET detectors would need a factor
of ten better laser power stability of roughly $\rm{RPN} = 3\!\times\!10^{-10} /
\sqrt{\rm{Hz}}$. 


{\bf High power laser}\\
 It is likely that the 700\,W laser power for ET-HF will be generated by a
 coherent combination of several high-power laser-amplifier stages seeded by one
 or more low-power low-noise master laser(s). Two different concepts are
 currently under investigation for such master-oscillator power-amplifier (MOPA)
 stages at the 250\,W output power level. One concept is based on
 mode-selectively pumped Nd:Vandat amplifiers which do not suffer from
 depolarizaton problems as Nd:YAG systems would do. A power of roughly 200\,W
 was lately generated with a commercial neoVAN-4S-HP amplifier chain with low
 noise and high spatial purity \cite{Bode2020, Willke2019}. Investigations are
 underway to increase the output power of such a solid state amplifier chain to
 250-300\,W. As solid-state MOPA chains are more complex compared to fiber based
 MOPAs, solid-state MOPAs serve as fall back solutions for ET-HF HPLs and will
 not be further discussed in this document.
 
Fiber amplifiers offer a highly-efficient and compact way to amplify laser light to the kW level. The amplification of narrow linewidth single mode seed lasers is, however, limited by nonlinear effects such as stimulated Brillouin scattering (SBS). Large mode-area (LMA) fibers in combination with losses for high-order spatial modes introduced by bending of the fiber can lead to high SBS free output powers in a single spatial mode operation. Sophisticated fiber designs (photonic crystal fibers \cite{Limpert2006}, photonic band gap fibers \cite{Gu2014} and chirally-coupled-core fibers \cite{Ma2014}) can guide a single mode with a large mode field diameter even without such bending losses.
The SBS threshold can be further increased by a differential temperature induced shift of the SBS gain spectrum along the fiber and via a counter-propagating pumping scheme. Several high power fiber amplifier systems that meet the demanding ET-HF HPL noise requirements have been demonstrated up to power levels of 300\,W \cite{Theeg2012, Zhao2018, Wellmann2019} .
%A 300\,W fiber amplifier system that meets the high spatial purity and low noise requirements was demonstrated by Theeg et al. \cite{Theeg2012, Wellmann2019} and a low noise 100\,W fiber amplifier system was reported by Zhao et al. (\cite{Zhao2018}). 
Higher power levels generated with fiber based MOPAs were reported in literature but only limited information on spatial purity and noise performance of these systems is available. Especially informaiton on the relative power noise (RPN) at radio frequencies is missing which is prone to increase at pump power levels well below the onset of significant SBS related power in the back-ward propagation direction \cite{Zhang2005}. Hence a conservative approach is taken for ET-HF that assumes that several fiber based MOPA systems will be coherently combined to form the ET-HF HPL. 
These fiber amplifier modules will each incorporate a mode-field adapter, a pump-light stripper, an active fiber and a pump-light combiner that couples the light of the fiber based pump diodes into the active LMA fiber. The amplified light will leave the fiber via a fused silica end cap to reduce the light intensity at the glas air interface and with this the risk of contamination induced damage.

Different options are under investigation for the seed laser design. A
fiber-oscillator in combination with a fiber-preamplifier allows for an
monolithic all fiber design that includes fiber based components such as Faraday
isolators (FIs), electro-optical modulators (EOMs) and acousto-optical
modulators (AOMs). All these components including the high power LMA amplifier
are spliced together and form a monolithic HPL module. One disadvantage of this
concept is, that modulators can only be used between the seed and the
preamplifier due to limited power handling capabilities of fiber modulators.

A second concepts relies on the non-planar ring-oscillator (NPRO) seed as used
in all currently operating GWDs. Free space EOMs, AOMs and FIs can be used to
condition the laser light before it is coupled into either a solid-state or a
fiber pre-amplifier. This amplifer is either spliced or free space coupled to
the mode-field adapter of the LMA high power amplifier. A trade-off study
between these two concepts will lead to the final ET-HF seed laser concept. Both
seed laser concepts can provide  actuators with enough range and speed for the
PSL frequency stabilization. As the power noise of a MOPA system is usually
dominated by the high power amplifier and as the modulation of the pump diodes
of this stage offer a large actuation range without cross coupling to the laser
frequency, the pump current of the LMA amplifier's pump diodes will serve as the
main PSL power actuator.

The coherent combination will be performed on a beam combiner (beam splitter)
before the pre-modecleaner cavity (PMC, see next subsection) . Both of the to-be
combined beams are separately aligned to the Eigenmode of the PMC which
guarantees an optimal spatial overlap. The differential phase between the beams
will be sensed at the second beam combiner port. A phase-lock control loop will
feed back to either one of the seed lasers (in case different seed lasers are
used) or to a piezo-controlled mirror or a fiber strecher in one amplifer's beam
path (in case of a single seed laser). Long term test will reveal if alignment
control loops are required to keep a good interference contrast at the beam
combiner.

Several commercial seed laser operating at 1550\,nm and Erbium based fiber
amplifiers for the same wavelength are available for use in ET-LF. The seed
lasers are either based on fiber oscillators or external-cavity diode lasers to
generate low noise beams with several 10\,mW power. Several MOPA configurations
with Er fiber amplifiers were tested at the 2\,W level and show spatial purity
and noise levels consistent with ET-LF requirements \cite{Meylahn2019}. The
amplification to the required power level of 5\,W by a second Er fiber amplifier
is straight forward. (Low noise fiber amplifiers with output power of more than
100\,W have been demonstrated \cite{deVarona2017}). In the case of a laser diode
based seed laser the pump current can be uses as a fast frequency actuator with
50\,kHz actuation bandwidth. The power noise can be reduced by feed-back to
either the fiber amplifier's pump diodes or via an external electro-optical
amplitude modulator (EOAM).

{\bf Prestabilization}\\
Even though laser systems with very low free running power and frequency noise
are chosen for ET sophisticated nested stabilization schemes are required to
achieve stability levels in the interferometers that do not limit the GW
sensitivity. A first stabilization layer, the so called prestabilization is
performed on the laser table outside the vacuum system. The goal of the
prestabilization system is to reduce the laser fluctuations well below the level
of noise added by the Doppler and beam pointing effects due to motion of the
laser table with respect to the seismically isolated interferometer frame.
Power in higher order spatial modes as well as beam pointing fluctuations are
reduced by passive spatial filtering with stable optical ring resonators called
modecleaner (PMC in case of the filtering on the laser table).  In the case of
ET-LF a fiber could be used as a spatial mode filter and as a tranfer fiber to
deliver the laser light via a seismically isolated output coupler in the
interferometer reference frame. This could strongly reduce the noise introduced
between the laser table and the suspended modecleaners. Further investigations
will reveal if non-linear effects or added phase noise in such a fiber would
prevail the benefits. In the case of ET-HF the power levels are to high for a
fiber based modecleaner such that a PMC needs to be part of the PSL. A PMC has
the additional benefit, that it filters power noise at rf frequencies and that
it can provide spatially stable pick-off ports for the frequency and power
stabilization and potentially for phase look loops of the squeezing or length
and alignment control subsystem.

The PSL frequency stabilization will use a rigid spacer high Finesse reference
cavity which is seismically isolated inside a vacuum system. A feedback control
loop with a high unity gain frequency of several hundred kHz is required to
allow for high bandwidth second layer control loops. These will use the
suspended modecleaner and the main interferometer as frequency references and
will feed back either via a summation point into the error point of the PSL loop
or to an AOM frequency shifter placed between the main laser and the rigid
reference cavity.

A high unity gain frequency is as well required for the PSL power stabilization
control loop. This needs to reduce the relative power noise (RPN) of the beam
leaving the PMC to roughly $10^{-8} / \sqrt{\rm{Hz}}$. As the speed of the
feedback to the diode laser pump source of the HPL is limited to several 10\,kHz
a fast EOAM after the seed laser or the pre-amplifier will be part of this
control loop. The power noise sensor for the power stabilization loop will be a
high-power photodiode placed into one of the pick-off ports of the PMC. Due to
pointing-to-RPN or polarization-to-RPN coupling by the suspended input
modecleaner a sensor for the second layer power stabilization has to be places
after the modecleaner close to the power recycling mirror. The second layer loop
power stabilization loop will either feed back into the error point of the first
loop or to an in-vacuum EOAM to achieve an RPN in the $10^{-10} / \sqrt{\rm{Hz}}$ range. Depending on the final RPN requirements sophisticated
power noise sensing schemes based on multi-photodiode arrays\cite{Kwee2009}, the
optical AC coupling technique \cite{Kaufer2019} or squeezed light assisted power
noise sensing \cite{Vahlbruch2018} might be required for this second loop. 

The general PSL layout will be very similar for the ET-HF and ET-LF PSLs. Due to
the two orders of magnitude lower power level the stabilization of the ET-LF
laser might be easier as more integrated fiber components can be used. These
have typically higher bandwidth and are less alignment sensitive. This small
advantage might, however, be compensated by the fact that the ET-LF requires
stability at much lower Fourier frequencies. This is generally harder as
scattered light and beam pointing coupling to the control loop's sensors is
larger at low frequencies. 

Prototypes for both the ET-HF and ET-LF PSLs will be set up in research labs in
the near future to test the ET laser designs, the stabilization concepts and to
gain insight into the longterm drift behavior of such systems.
 