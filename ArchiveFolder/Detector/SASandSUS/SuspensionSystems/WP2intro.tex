
%\documentclass[color,cite,epsfig,12pt]{article}
%\usepackage{graphicx}

%\begin{document}


\FloatBarrier
\subsection{Description}
%
% put here the description of the content of the section
%
%\emph{
%Responsible:  WP2 coordinator, F.\ Ricci\\
%}

The suspension system of a gravitational wave detector is conceived to isolate the test
masses from ground seismic vibrations and local disturbances so that they are considered 
nearly in a free fall state in the frequency bandwidth of interest. 
A typical suspension chain is obtained cascading a set of passive mechanical filters
providing a suited attenuation from seismic and acoustic noise above a certain cut-off
frequency. Below this frequency value an active feed-back control strategy is developed 
by using sensors and actuators disseminated along the suspension chain and keeping the 
interferometer at its working point.
  
The upper part of the suspension chain, called \emph{Superattenuator} (SA), is essentially a 
N-stage pendulum supported by a three-leg elastic structure, called Inverted Pendulum (IP). 
In an N-stage pendulum the horizontal displacement of the suspension point, at a frequency 
$f$ much higher than its normal modes $(f_1, f_2, ........, f_N)$, is transmitted to the 
last stage with an attenuation proportional to $f^{-2N}$.
In order to attenuate the vertical vibrations, each mass of the multi-stage pendulum is 
replaced by a cylindrical mechanical filter with a set of concentric cantilever blade 
springs with low stiffness. The blades support, through a long metallic wire, the next 
mechanical filter, forming a chain of low frequency oscillators also in the vertical direction. 
In Virgo the blades work in parallel with a magnetic anti-spring system, assembled 
on each filter and designed to reduce its vertical mode from about 1.5 Hz down to 0.5 Hz. 
Thanks to the anti-springs the vertical modes of the chain are all below 3 Hz. 
Starting from a few Hz, the mirror is thus well isolated from vertical seismic noise too. 
 
The last suspension stage (LSS), called {\it {optical  payload} } or simply 
{\it {payload}}, is the system designed to couple the test mass to the SA chain, 
to compensate the residual seismic noise and to steer the mirror through internal 
forces exerted from the last SA element. The main components of a Virgo-like LSS are: 
the Marionette, the Recoil Mass (RM) and the Mirror. The marionette is the first 
stage used to control the mirror position by means of coil-magnet actuators acting 
between the last stage upper part suspension and the marionette arms, while the recoil 
mass is used to protect and to steer the mirror. On its mechanical structure, indeed, 
the coils of the electro-magnetic actuators acting on the mirror back side are mounted.

\noindent
LSS plays also another important role: all the mechanical elements which are connected 
to the mirror are designed not to degrade the intrinsic mechanical losses of the 
mirror itself. This is necessary because of the well known relation between mechanical 
dissipations and thermal motion in macroscopic systems. The thermal noise contribution 
is reduced by developing sophisticated suspension systems with materials having low 
mechanical dissipation and low friction mechanical clamps. However, to achieve  
the goal sensitivity of the ET Project at low and intermediate frequency, 
a further reduction of thermal noise contribution is obtained by cooling the mirror 
and its last stage at cryogenic temperature\cite{ricci_moriond}. Cryogenic operation 
introduces additional difficulties, but the benefits in thermal noise reduction 
is also enhanced by the materials selection with improved properties at low 
temperatures. 

\noindent
A major challenge for cooling the mirror is to provide an efficient path for heat 
conduction while still maintaining good thermal noise and mechanical isolation performance. 
The cooling system needs to provide adequate heat extraction from the test masses, 
for both steady state operation and for cooling from room temperature in a reasonable 
time, without adding noise or short-circuiting the mechanical isolation. This constraint 
has an important impact on the design of a last stage of cryogenic suspension.

In the next sections we analyze in sequence all these aspects and we will try to summarize 
a perspective solution for each of them presenting also possible alternative approaches. 
Finally, an overview of the related technologies to be developed is presentsd in section 4.6.

%
\FloatBarrier
\FloatBarrier
\subsection{Executive summary}
%

We organized the design study of the ET suspension focusing the activity on four main arguments:
\begin{itemize}
\item {The upper part of the suspension, the \emph{superattenuator} (SA),  providing 
the needed seismic and acoustic isolation. This issue is discussed and the solution 
is presented in the Upper Stage section~\ref{sec:Upper_stages_mechanics}; }
\item{The last stage (payload), crucial for the defining the thermal noise 
performances and the mirror control are presented in paragraph~\ref{sec:last_stage};}
\item{The local control strategy and the related instrumentation, analyzed in 
sections~\ref{sec:local_control}, \ref{sec:em_actuator}, \ref{sec:electrostatic};}
\item{An overview of the main technologies to be developed for achieving the ET 
scientific goal (see sec.~\ref{sec:crystal-production}, \ref{sec:bonding}, 
\ref{sec:accelerometers}).} 
\end{itemize} 

We notice that to clarify some of the open problems, several R\&D programs have been 
carried out with limited resources. In particular, a bigger effort should be done 
in the domain of the material studies and of the low temperature  sensing devices.  
Nevertheless we succeed in preparing a reference solution and a set of alternative 
scenarios for driving the ET technical design. However the solution selection 
requires intensive experimental activity and the coordination efforts will be  
significantly effective only if continuous activity in the laboratories is sustained 
and coordinated in collaboration with the Japanese project LCGT.

The preliminary activity of the ET working group, devoted to the conceptual design 
of the suspensions, was centered on the definition of ET suspension requirement. 
In particular we collected and compared the mechanical and thermal properties of 
the materials at room and low temperature. 
Both silicon and sapphire potentially offer superior performance at cryogenic temperatures. 
In particular silicon samples can be obtained in large cylindrical blocks of mass higher 
than 100 kg. Its attractive thermal and thermo-mechanical properties makes it a strong 
candidate for the mirror and the suspension fibre in the future detector operating 
at cryogenic temperatures. 

\medskip

The low and high frequency interferometers ( ET-LF and ET-HF) proposed in the ET 
design study require different suspension systems. In the following we will show 
that the SA as it is operating within the Virgo interferometer is already compliant 
with the HF interferometer requirements. On the other hand for  ET-LF a dedicated 
effort is required. In fact, to extend the detection bandwidth of the Einstein 
Telescope in the low-frequency region starting from a couple of Hz, a better seismic 
attenuation in the ultra-low frequency range is needed. 
The \emph{Superattenuator} dynamics in the low frequency range, where the inner 
normal modes of the mechanical filter chain are confined, has been simulated by 
using the electro-magnetic equivalent circuit and a detailed simulation campaign 
devoted to this design study has been performed. On this base we concluded that 
we can fix to six the number of filters because a better attenuation performance 
in the high frequency range is not necessary anymore since the safety margin is 
large enough in Advanced Virgo and even larger in an underground environment. 
Then, a SA 17-m high with six magnetic anti-spring filters ("equal-spaced" 
configuration) tuned with a vertical cut-off frequency around 300\,mHz, represents 
the reference solution for  ET-LF interferometer. 

\medskip

As we anticipated above, the last stage of the suspension system is crucial in defining 
the suspension thermal noise performances. In particular for the  ET-LF case we deal 
with a system of coupled oscillators with masses at different temperature. For this reason, 
the thermal noise evaluation required to develop a specific model,where
the interacting masses are at different  temperature in steady state condition. 
The study of the thermal noise 
of this system has been carried on by using two different methods and the suspension 
thermal noise curves have been compared with the ET-D goal sensitivity curve driving 
the design of the last stage of the mirror suspension. 

Thus, the conceptual design of the payload for LF and  ET-HF has been completed. 
It includes for  ET-LF the analysis of the heat flow, the definition of the fibre 
material and geometry and the heat path from the mirror to the cold box via the 
marionette trough the suspension fibres and the dedicated heat sinks.

\medskip

The  ET-LF payload will be installed in the lower part of the vacuum tower hosting 
the 17-m long super-attenuator chain. The vacuum tower basement is a cryostat separated 
to the upper part by a roof crossed by the Ti-6Al-4V thin rod, which holds the 
whole payload. The cold box is set on top of a cryo 
compatible attenuator chain hosted in an ancillary cryostat. The box will keep the mirror at cryogenic temperature: 
it can be either a simple liquid helium container in the case of a cryoplant based 
on cryofluids or the cold head of a cryo refrigerator in the case of a cryocooler 
plant. Both solutions are discussed and compared in more details in the infrastructure 
section. The final choice will depend on the outcome of a dedicated  R\&D effort.

\medskip

A sophisticated control systems will be implemented in order to hierarchically 
control test mass positions to bring the interferometer to operation.
The actuation system is needed to align the interferometer mirrors 
and to automatically maintain the operation point. It is based on the signals 
derived from the circulating light  to apply internal forces exerted 
through the last stages of the suspension. 
The actuation based on a coil-magnet system for ET is reviewed in details and 
its limit associated to the $1/f$ potential noise is assessed; the alternative 
electrostatic approach is also discussed.

\noindent
For the  coarse mirror control optical conduits and bundle fibre served as position sensors have been tested  at low temperature  \cite{VFC}. They can be used together with short-arm optical levers and small 
interferometric systems for fine position control to achieve an alignment accuracy within 
10 nrad RMS. Hence low frequency cryogenic accelerometers have being developed 
for this purpose.

\medskip

Finally, it should be underlined that the complex study on the ET suspension system 
is not limited to a definition of a reference solution. Some crucial technologies 
to be developed in the domain of the slicon fibre and cryo-device construction have
been identified. These are issues of potential interest for industrial world too.


 
% \end{document}>>>>>>> .r616
