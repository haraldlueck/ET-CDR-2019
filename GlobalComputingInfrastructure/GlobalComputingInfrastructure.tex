\chapter{Global Computing Infrastructure}
\label{chap:GlobalComputingInfrastructure}
As with most large experiments, the scale of ET computational challenges is such that the computing infrastructure architecture cannot be solely based on local resources. The physics community at large has a long experience in designing, building and managing global-scale computing infrastructures that can be exploited to cater to these needs.  
This section deals with the challenges of such an infrastructure dedicated to the search for GW sources with the ET. 
A brief description of examples of existing search methods and parameter estimation tools for different sources is given. For these methods, we provide a some considerations about the computational resources needed. 
Subsequently, we try to estimate the impact of upcoming computing technologies on ET data analysis, and evaluate a possible computing infrastructure architecture.

\section[Computing challenges]{Computing challenges}
\label{sec:Computing challenges}
\section[Technology outlook]{Technology outlook}
\label{sec:Technology outlook}
\subsection{Hardware}
\subsection{Software}
\section[Distributed computing infrastructure]{Distributed computing infrastructure}
\label{sec:Distributed computing infrastructure}
During the Advanced Detectors Era, distributed computing activities were mostly driven by LHC requirements, that led to the design and deployment of the WLCG infrastructure.
Within the timeframe of the ET initial phases, several other collaborations will reach LHC-like scales both in data sample size and computing power requirements: SKA and CTA, for example, but also outside of physics with, for example, the Human Brain project. Furthermore, High Luminosity LHC runs will increase both its data volume and computing requirements by large factors
Thus, it is highly likely that a large scale shared computing infrastructure will be built to cater to the needs of all such collaborations. 
\section[Computing strategies]{Computing strategies}
\label{sec:Computing strategies}